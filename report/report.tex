\documentclass[runningheads,a4paper]{llncs}
\usepackage{amssymb}
\setcounter{tocdepth}{3}
\usepackage{graphicx}
\usepackage{amsmath}
\usepackage{caption}
\usepackage{subfig}
\usepackage{csvsimple}
\usepackage{multirow}
\usepackage{array}
\usepackage{booktabs}
\usepackage{url}
\graphicspath{{./images/}}

\begin{document}

\title{Feature Representation of Point Clouds as 2D Perspective Rasters}
\author{Leland Jansen \and Nathan Liebrecht}
\date{April 2019}
\institute{Deptartment of Computing Science, University of Alberta, Canada}

\maketitle

\begin{abstract}
As three-dimensional point clouds become an increasingly popular way to
represent spatial data, with it comes a need to classify objects represented in
this form. This paper proposes an image feature representation technique which
classifies 3D point clouds by representing projecting them in a variety of
perspectives in two-dimensional space, rasterizing these projections, then using
image classification techniques to classify the rasters. Existing work has had
limited success using this technique with only one perspective, likely due to
the considerable information loss during the projection. We hypothesize that
multiple projections will mitigate these effects by capturing lost data in other
perspective projections. Our method proves to be effective when
classifying pedestrians and vehicles in the Sydney Urban Objects Dataset, where
our technique has achieved a classification accuracy of 98.42\% and an f1 score of 0.9843.
\end{abstract}

\section{Introduction}
Point clouds are a common, primitive representation of three-dimensional (3D)
data. Creating these point clouds can be achieved using a variety of
technologies such as LiDAR imaging systems. As the technology becomes more
available, a need arises to quickly and accurately extract features from these
data to perform tasks such as object recognition. For example, point cloud
object recognition is especially relevant in autonomous vehicle research to
detect other cars, pedestrians, road signs, etc. and inform real-time decisions.
Classifying objects represented by 3D point clouds is therefore a highly
relevant problem to solve. Furthermore, solutions for self-driving cars must run
in real-time and thus must be highly performant.

Our research proposes representing 3D point clouds as a series of 2D rasterized
images generated from different perspectives. These images can then be joined
into a single raster and fed into a one-shot detection network such as YOLO.
This will maintain high efficiency while providing semi-redundant perspectives
of the original point cloud.

\subsection{Importance of the problem}
Three-dimensional point clouds including those retrieved from LiDAR imaging
systems are gaining popularity as the technology becomes more available. This is
especially true in autonomous vehicle research which requires cars to detect
pedestrians, other cars, road signs, etc. Classifying objects represented by 3D
point clouds is therefore a relevant problem to solve.

One of the challenges faced by autonomous vehicles is the high computation power
requirements to achieve real-time operation. This adds significant cost to the
car and its development infrastructure. Therefore, both accuracy and performance
are essential to have a safe, reliable vehicle that responds in real-time.

\section{Related Work}
Present work is focused on classifying 3D point clouds directly from the raw
data, not through 2D perspective rasters. Additionally, a substantial research
has been performed to classify objects through 2D camera images. 

This technique opens up the possibility of using any of the many existing image
classification algorithms possibly leading to classification accuracy gains.

Huang and You \cite{huang2016point} call out that a simple 2D representation of
a point-cloud can lead to a loss in 3D structural representation. Here we hope
to avoid this loss of information by adding additional rotation to the scene
and detecting many instances at once.

Landrieu and Simonovsky \cite{landrieu2018large} represent the state-of-the-art
for point-cloud semantic segmentation. This paper uses a technique dubbed
``superpoint graph'' which works in a similar way to superpixels; that is, by
grouping neighboring similar clusters of points into larger objects which
simplifies further processing by reducing the amount of data. In a similar way
we hope to reduce the total amount of data by flattening the 3D space into a
more consumable 2D representation.

Substantial research has been performed in the space of 2D image object
recognition. Early techniques included feature detectors such as SIFT and SURF
which extracted invariant features from images and were then used to match to
feed into a model such as a database of features or an SVM. Recently, deep
learning has proven to be the de-facto standard for 2D image object
classification. Techniques such as convolutional networks are often used. These
have been shown to be both highly accurate and performant.

Research has also been performed in the space of 3D point cloud feature
recognition. Early research employed techniques such as surface reconstruction
and feature extraction such as surface normals and gradients. Recent research
has explored promising techniques such as directly ingesting 3D point clouds
into deep neural networks using a voxelized structure. However this approach can
be challenging to implement efficiently on hardware since 3D space is quite
sparse and points can exist in continuous space rather than in a discrete grid
like traditional images.

\section{Novelty}
We hope to improve current techniques by combining the rich 3D representation of
objects using point clouds with the great efficiency of processing 2D images. By
using a one-shot detection network such as YOLO, we can maintain high efficiency
while adding semi-redundant rotated 2D perspectives of the point-clouds. We may
also increase information density by encoding point depth information in a
colour channel. For example, further pixels could be darker. We hope this shaded
depth representation will achieve greater edge discrimination.

As described below, we have chosen Sydney Urban Objects Dataset to evaluate our
technique. Maturana and Scherer \cite{maturana2015voxnet} report an average F1
classification score of 0.73 using a convolutional neural network. Their methods 
also take around 6ms to classify an object using a Tesla K40 GPU. We
will investigate potential accuracy and performance differences using our perspective
rasterization techniques coupled with a one-shot classifier.

\section{Data Set}
We have chosen the Sydney Urban Objects Dataset for our project due to its size
and the variety of objects. As part of this project we may chose to augment this
dataset by introducing small amounts of noise and distortion to the point-cloud.
Figure \ref{fig:data-example} shows sample point clouds from this data set.

\begin{figure}[h]
  \caption{Example objects in our chosen dataset}
  \centering
  \includegraphics[scale=0.1]{data-example}
  \label{fig:data-example}
\end{figure}

\section{Methodology}
\subsection{Preprocessing}
Our proposed technique involves the following pre-processing steps.
\begin{enumerate}
  \item Load the dataset into memory by parsing the raw data file.
  \item Set the scene properties such as background color, point color, etc.
  \item Determine the geometric centre of the point cloud.
  \item Position the “camera” such that it is facing the centre of the point
    cloud. Presently, the radius is manually determined and hard-coded such that
    the entire point cloud is viewable by the camera. Future work could involve
    automatically adjusting the radius such that the entire point cloud is fully
    captured by the camera.
  \item Move the camera around the center of the point cloud (maintaining the
    same radius) to capture 64 perspectives of the point cloud.
  \item At each iteration, take a “picture” and save it to as a temporary file
    on the filesystem.
  \item After all perspectives have been captured, stitch the images together in
    a 8x8 grid.
  \item Reduce the resolution so the image size is more reasonable.
  \item Save the stitched image.
  \item Generate a labels file with 64 bounding box coordinates, one around each
    image in our grid.
\end{enumerate}

To move the camera around the image, we translate the camera’s position from
cartesian to polar coordinates as described in Equations
\ref{polar-coordinates:r}-\ref{polar-coordinates:phi} and visualized in Figure
\ref{fig:polar-coordinates}.
Additionally, we ensure that the camera is “pointing” at the origin.

\begin{align}
  \label{polar-coordinates:r}
  r &= \sqrt{x^2 + y^2 + z^2} \\
  \label{polar-coordinates:theta}
  \theta &= \tan\left(\frac{y}{x}\right) \\
  \label{polar-coordinates:phi}
  \phi &= \cos\left(\frac{z}{r}\right)
\end{align}

\begin{figure}[h]
  \caption{Polar and cartesian coordinates}
  \centering
  \includegraphics[scale=0.15]{polar-coordinates}
  \label{fig:polar-coordinates}
\end{figure}

An example result of a car’s point cloud after preprocessing is shown in Figure
\ref{fig:example-preprocessing}. Note that we have inverted the colours for greater
visibility when printed.

\begin{figure}[h]
  \caption{Example objects in our chosen dataset}
  \centering
  \includegraphics[scale=0.15]{example-preprocessing-car}
  \label{fig:example-preprocessing}
\end{figure}

\subsection{Challenges}
We encountered a major difficulty working with pptk. The client Python library
communicates with a C++ backend via a socket. However, no effort is made to
synchronize the communication or signal when the socket communication is
complete. Therefore, after a considerable amount of debugging, we discovered
that we must sleep ~0.2 seconds between rendering and capturing the image to
save to the filesystem. This makes the data rendering process extremely slow.

\subsection{Training}
We train our preprocessed images on the YOLOv3 network \cite{redmon2016you} which 
has an excellent reputation for high performance and accurate object detection. 
Simply, we add 64 bounding boxes, one for each perspective, and feed this ground 
truth data to the YOLOv3 network. Figure \ref{fig:training-600} shows the progression
of training on a 1080 Ti GPU over the span of around 6 hours.

\begin{figure}[h]
  \caption{Training with dataset of 630 images (70/30 training/testing split)}
  \centering
  \includegraphics[scale=0.15]{training-600}
  \label{fig:training-600}
\end{figure}

\subsection{Inference}
Once we have preprocessed the point cloud and arranged it into an image grid, we
then perform inference on this image using the YOLOv3 network. The result of this
inference step is a list of classes and bounding boxes. To make sense of these
detections, we first sort the list of detected classes. We then count the number
of detections for the most detected class and apply a configurable threshold
to this number. If the number of detections equals or exceeds the given threshold,
we consider this class to be the final classification. Using this voting
mechanism we are able to add redundancy through multiple perspectives.

\subsection{Future Work}
\subsubsection{Tighter Bounding Box}
We presently draw the the image label bounding box around the entire perspective
raster which includes a considerable amount of “empty” space (i.e. black
background). We might see better classification results with a tighter bounding
box around each image. Using OpenCV looks like a promising library to achieve
this.

\subsubsection{Encode Additional Information Through Colour}
Right now each point in the point cloud is a single, white dot. We might see
accuracy improvements encoding additional information into the colour of each
point. For example, depth information such as distance from the camera could be
encoded into one colour channel, curvature into another channel, etc.

\subsubsection{Scale Robustness with Multiple Radii}
All perspective rasters are presently captured using a single, fixed radius
from the geometric center of the point cloud. Also varying the radius of the
camera from the point cloud’s center might make our classifier more robust to
scale. 

\section{Results}

\begin{figure}[h]
  \caption{Example detection failure: building classified as a car}
  \centering
  \includegraphics[scale=0.15]{example-failure}
  \label{fig:example-failure}
\end{figure}

\subsection{Accuracy}
As seen in the ``Training" section, the mAP is around 0.70 which is rather low. 
However by using our voting mechanism we are able to acheive a far greater aggregate
result. For the Sydney testing dataset the highest accuracy we have achieved is 98.42\%.

\subsection{Accuracy vs Threshold}
As mentioned in the ``Inference" section, an important parameter to tune is the threshold
of detections at which we consider the object to be classified. We perform an empirical
study to guage the most effective value and find that the optimal value for us is 58.
Figure \ref{fig:accuracy-vs-threshold} shows the relation between the threshold value
and our accuracy.


\begin{figure}[h]
  \caption{Accuracy vs Threshold}
  \centering
  \includegraphics[scale=0.15]{accuracy-vs-threshold}
  \label{fig:accuracy-vs-threshold}
\end{figure}


\subsection{Evaluation}
We perform a standard classfication evaluation which consists of finding the precision,
recall, and f1 values for each class, and the weighted average of these values over each
class. Our results are shown in the following table: 

\begin{center}
\begin{tabular}{ |c|c|c|c|c|c| } 
\hline
	    & & precision & recall & f1-score & support \\
\hline
\multirow{3}{4em}{Class} & car & 0.9839 & 1.0000 & 0.9919 & 61  \\
  		  & pedestrian & 1.0000 & 0.9688 & 0.9841 & 96  \\
      		      & none & 0.9429 & 1.0000 & 0.9706 & 33  \\
\hline
\multirow{3}{4em}{Average}  & micro avg & 0.9842 & 0.9842 & 0.9842 & 190 \\
   			    & macro avg & 0.9756 & 0.9896 & 0.9822 & 190 \\
			 & weighted avg & 0.9849 & 0.9842 & 0.9843 & 190 \\

 \hline
\end{tabular}
\end{center}

In order to evaluate the run-time of our system, we simply take the total runtime of the inference
step and divide by the number of images. Using a 1080 Ti we are able to get 
inference times of approximately 10 milliseconds per object. Because an 
optimized preprocessing step (rasterization of simple points) would be 
almost immediate using modern graphics acceleration hardware, we can assume 
that the majority of the time will be spent in the inference step and that 
this time meaured is a good estimate of the end-to-end time spent from lidar 
point-cloud to classification.

\subsection{Comparison with Existing Methods}
We found two papers that attempt to classify the point cloud data in our chosen dataset.
In De Deuge et al. Unsupervised Feature Learning for Classification of Outdoor 3D Scans
\cite{de2013unsupervised},
they acheive a maximum f1 score of 0.671. In Quadros, Representing 3D shape in sparse 
range images for urban object classification \cite{quadros2013representing}, they were 
able to attain an f1 score of 0.7. The average f1 score of 0.9843 that we attained 
indicates a significantly higher performance than these contemporaries. In particular,
we have much better f1 score in vehicle classification where Quadros, 2014 achieves an f1
score of 0.77. Since the Sydney dataset is quite limited by the standards of today's
deep learning needs, we expect that more data would greatly increase the performance of
our method.

In terms of run-time, previous papers are more effecient, however this comes at
the expense of accuracy.


\section{Conclusion}
In conclusion, we are able to effectively classify vehicle and pedestrain point-clouds in the 
Sydney Urban Objects Dataset by taking advantage of recent breakthroughs in 2D image
detection technologies. By rasterizing multiple 2D perspectives of the point-clouds and
introducing a voting mechanism to resolve multiple detections we are able to preserve
depth information which reducing the effective dimensionality of our input.

\bibliography{report}
\bibliographystyle{ieeetr}

\end{document}
